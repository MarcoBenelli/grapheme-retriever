\chapter{Conclusioni}\label{ch:conclusioni}

Quello che abbiamo fatto \`e stato, a partire da digitalizzazioni di documenti manoscritti medievali, trovare un metodo per l'estrazione di alcuni grafemi che ci interessavano.

Prima di tutto abbiamo binarizzato le immagini per rendere pi\`u semplici le operazioni successive. Dopodich\'e abbiamo utilizzato metodi basati su proiezioni orizzontali e verticali per suddividere il documento nelle righe e nei caratteri che lo compongono.

Una volta isolate le varie immagini dei caratteri e estratti da essi i grafemi mediante lo studio delle componenti connesse, siamo passati a confrontarle con dei modelli estratti manualmente in precedenza per trovare quelle che rappresentano ciascuno dei grafemi che ci interessano.

Il confronto avviene mediante vari passaggi successivi. Inizialmente si trasformano le immagini per renderle tutte delle stesse dimensioni. Questo avviene secondo varie modalit\`a che verranno confrontate fra di loro. Infine si confrontano le immagini tramite l'indice di Jaccard.

Per confrontare la bont\`a delle varie modalit\`a si usano degli scan gi\`a analizzati manualmente. I risultati degli esperimenti sono riportati nel capitolo \ref{ch:experiments}, da cui si vede che tutte le modalit\`a si comportano in maniera molto simile, con piccole differenze.

Come ultima cosa si \`e usata un'interfaccia interattiva per rendere il programma migliore in situazioni in cui non basta un unico modello per analizzare tutto il documento. Questo \`e stato descritto nel capitolo \ref{ch:interactivity}, in cui si pu\`o vedere il miglioramento considerevole rispetto alla modalit\`a non interattiva.

In conclusione, il programma risulta utile per la ricerca dei grafemi che ci interessavano e si rivela anche versatile tramite l'uso della modalit\`a interattiva, nonostante l'apparente semplicit\`a delle tecniche utilizzate.